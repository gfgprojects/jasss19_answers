\documentclass[a4paper,12pt]{article}
%\usepackage[italian, english]{babel}
%\usepackage[T1]{fontenc}
%\usepackage[utf8]{inputenc}
%\usepackage[italian, english]{babel}
\usepackage{graphicx}
\usepackage[round]{natbib}
\usepackage[table]{xcolor}
%\usepackage{wrapfig}
\usepackage{rotating}
%\usepackage{import}
\setlength{\oddsidemargin}{0pt}
\addtolength{\voffset}{-52pt}
\setlength{\textwidth}{6.27in}
\setlength{\textheight}{10.in}
\usepackage{setspace}
\usepackage{url}
\onehalfspacing



% * <edidigiu@gmail.com> 15:58:59 01 Mar 2019 UTC+0100:
% Sembra user friendly e ci sono i commenti, altri vantaggi rispetto ad overleaf?

\begin{document}
	
	\begin{center}
		\textbf{JASSS}

	\vspace{4mm}
\textbf{Article reference 2018:74:1}

	\vspace{4mm}
	{\Large\textbf{Changes to the manuscript on the basis of Reviewers' comments}}

\vspace{4mm}
\textbf{Manuscript title:}

\textbf{Investigating the effects of Climate Shocks on the Wheat International Market with an Agent-Based Model: The 2010 Russian Federation Case}
	\end{center}
	%\uppercase{Reply to Reviewer 1}
\vspace{10mm}


We thank the reviewers for the accurate and qualified revision of the
manuscript. When we read their reports, we were particularly pleased by their ability to
understand the exact purpose of the paper. Their comments showed us the weaknesses of the draft and signaled the points which needed improvements and more careful consideration from our side. 
We worked to strengthen the identified weaknesses and we provide a revised version of the paper.

\iffalse
First of all, their comments convinced us to change the title of the manuscript from \textit{"Investigating the effects of Climate Shocks on the Wheat International Market with an Agent-Based Model: The 2010 Russian Federation Case"} to 
\textit{"A computational model of the Wheat Global Market: the 2010 Russian Federation Case"}.
\fi

\iffalse
First of all, we would like to thank the reviewer for the accurate and qualified revision of the
manuscript. When we read the report, we were particularly impressed by the reviewer's ability to
understand the exact purpose of the model reported in the manuscript.
Her/his comments let us see the weaknesses that escaped our eyes. 
We worked to strengthen the identified weaknesses and we provide a revised version of the paper that we hope the reviewer will find significantly improved.
\fi


A detailed description of the changes made on the basis of  Reviewers' comments and suggestions is provided hereafter. 


%Below, you will find a description of how the major comments were addressed in the manuscript. Original reviewer

\vskip1cm
\textbf{Reviewer 1}




\begin{itemize}
\item The reviewer asks to qualify to which degree the model is an ABM. 

Although the model has agents, its present implementation does not follow the standard ABM approach. 
The present version of the model is an attempt to create a virtual world with a limited number of macro agents that can be related to real-world countries or groups of countries. The aim is to forge each agent as close as possible to the real world counterpart and to investigate the dynamics resulting from their interaction. In this sense, the current implementation is at the crossroad between ABM and Microsimulation approach. 
Considering the reviewer helpful comment, we have decided to refer more generally to a computational model. The title was also modified accordingly.
\item The model is largely presented in a condensed and itemized form which is hard to follow.

    We have worked to have a more intuitive and less technical presentation of the model. As also recommended by reviewer 2, we have reworked the model description to make it suitable for a broader audience. The reader is now pointed to the documentation for the details. 
    
    The specific points raised by the reviewer concerning section 2 were taken into account to rework the section.

Concerning figure 12 (figure 16 in the new version), the price with and without ban are exactly the same up to 2010. After the ban, the gap between the two series start growing. 
Unfortunately we have only two years of data after the ban and the reader may fail to focus on comparing data from these two years. To make it happen we added the following sentence ``world price would have been significantly lower in 2011-2013 \ldots''. Moreover we supply an explanation for constant prices at the end of the time series stating that  ``the model outputs also the projection of prices for the following years. They are obtained under the simplification that produced and demanded quantity keep constant to the 2013 levels'' (paragraph 4.18).   

\item A careful revision of the English language was performed by a professional proofreading service.


\end{itemize}

\vskip1cm
\textbf{Reviewer 2}
\begin{itemize}
	\item The model description is hard to follow and should be designed for a broader audience.

		We have reduced the technical details to give a more intuitive presentation of the model. 
		We have followed the useful reviewer's recommendation designing a graphical representation of the model structure and of the agents' decision-making process. This additional work is included in the new version of the paper by using a visual representation (Fig. 1), two diagrams (Fig. 2 and 3) and a practical example (Tab. 1). Moreover, we have given details about the functioning of the real wheat market and described how our model treats the corresponding factors. 
    
\item It is recommended to give more weight to the ``qualitative findings'' of the model.

The reviewer's comment made us realize that our effort in replicating average world prices brought a number of other qualitative results worth to be reported in the paper. In our opinion this has improved the insights obtained from the model. 
In particular, we realized that the qualitative results mainly concerns the quantities exchanged. Therefore, we redesigned the results section giving more space to the results on quantities. In our view, following the reviewer's recommendation we obtain the result s/he awaited: balancing the paper results section between the quantitative results on prices and the qualitative results on quantities. 
In our opinion, the most significant addition to the results section concerns the discussion on the Trade Network. We introduced in the new version of the paper the results of the long term analysis of the wheat trade network. It is reported in figure 11 which shows how the model replicates the most important real world commercial relationship. The dynamic analysis of the trade network, already present in the previous version, was better described in the new version. To overcome the difficulties in presenting dynamic analysis in a scientific article, we renew the invitation to visit \url{http://erre.unich.it/wheat_map/} where a number of short movie show the dynamics of the trade network.       


    \item ``A better analysis of the requirements of a model of the world wheat market would bring the scale of the problem in better balance with level of detail in the model. We recommend the authors to focus on this area of improvement'' 
    
    We have added a discussion and some associated references for a better description of the factors that are responsible for the volatility of wheat prices. For example, we have included arguments about one of the most significant factor of this market, which is the speculation (in Russian Case paragraph as well as in Conclusion and Discussion section). There, we also state that ``the two sharp peaks in 2007/2008 and 2010/2011 are faithfully reproduced also if they are specifically due to investor speculation that is not strictly accounted in our model implementation''. The subject of this answer is connected also to the following comment of the reviewer.
    
    \item ``improve the analysis on the robustness of their overall conclusion on the Russian ban''
    
	    We have provided an improved description of the Russian Federation ban declaration and a deeper discussion both of causes and effects of such a decision. As a consequence, we were able to reinforce the discussion on the results obtained {\color{red} either by the quantity or the price side}. In this sense, we have highlighted in paragraph 4.19 that  ``The price simulation without ban confirms the uselessness of the ban while the general price increase is due to net supply shock between what has been declared early in 2010 and what happened later (Wegren 2011; Adjemian et al. 2014). In this way, predictions of a poor harvest in the Russian Federation lead to dramatic increases in purchases and prices, even though this drop in production would not have a dramatic impact on global supply. This then triggers export bans in exporting countries, which in turn makes importers even more nervous and so generates a self-fulfilling prophecy (Welton 2011). This may also help to explain the persistence of high prices as highlighted in figure 16 with and without the ban.''. 

\item A careful revision of the English language was performed by a professional proofreading service.
	
\end{itemize}


\vskip1cm
We would like to thank the reviewers once more for the careful review and detailed comments as well as constructive suggestions. We hope the reviewers will find the new version of the paper significantly improved. We kindly ask them to indicate additional improvements they will find during the reading of this new version. 

\vskip1cm
We enclosed here a list of new references included in this version of the paper:

\begin{enumerate}
    \item S.K. Wegren (2011) Food Security and Russia's 2010 Drought,Eurasian Geography and Economics, 52:1, 140-156, DOI: 10.2747/1539-7216.52.1.140

    \item Opinion Global Economy, Time to regulate volatile food markets, Financial Times, August 3, 2010, p. 1

    \item OECD., 2011. Agricultural Policy Monitoring and Evaluation 2011: OECD Countries and Emerging Economies

    \item Tang K, Xiong W (2011) Index investment and financialization of commodities. NBER Working Paper 16385 (National Bureau of Economic Research, Cambridge, MA). Available at www.nber.org/papers/w16385.pdf.

    \item M. Lagi, Y. Bar-Yam, K.Z. Bertrand, Y. Bar-Yam, Accurate market price formation model with both supply-demand and trend-following for global food prices providing policy recommendations Proc. Natl. Acad. Sci., 112 (2015), pp. E6119-E6128
    \item Rutten, M., Shutes, L., \& Meijerink, G. (2013). Sit down at the ballgame: How trade barriers make the world less food secure. Food Policy, 38, 1–10.
    \item OECD-FAO Agricultural Outlook 2012, Biofuels. Available at \url{ http://www.fao.org/fileadmin/templates/est/COMM_MARKETS_MONITORING/Oilcrops/Documents/OECD_Reports/biofuels_chapter.pdf}
    \item Musunuru N (2014) Modeling price volatility linkages between corn and wheat: a multivariate GARCH estimation. Int Adv Econ Res 20(3):269–280
    \item Janzen, Joseph P., Carter, Colin A., Smith, Aaron D. Smith, and Michael K. Adjemian. Deconstructing Wheat Price Spikes: A Model of Supply and Demand, Financial Speculation, and Commodity Price Comovement, ERR-165, U.S. Department of Agriculture, Economic Research Service, April 2014.
    \item Adjemian, M.K., P. Garcia, S.H. Irwin, and A.D. Smith. 2013. Non-convergence in Domestic Commodity Futures Markets: Causes, Consequences, and Remedies, Economic Information Bulletin No. 115, U.S. Department of Agriculture, Economic Research Service, August, \url{www.ers.usda.gov/publications/}
    \item Garcia, P., S.H. Irwin, and A.D. Smith. 2012. “Futures Market Failure?” Working Paper, University of California, Davis.
    \item    Delincé, J., 2017. Recent practices and advances for AMIS crop yield forecasting at farm/parcel level: a review. FAO–AMIS Publication, Rome, \url{http://www.fao.org/3/a-i7339e.pdf} (accessed 10 November 2017).
M.E. Brown Remote sensing technology and land use analysis in food security assessment J. Land Use Sci., 11 (2016), pp. 623-641, \url{10.1080/1747423X.2016.1195455}
    \item    G20, 2011. G20 Cannes Summit Final Declaration “Building Our Common Future: Renewed Collective Action for the Benefit of All”, \url{http://www.g20.utoronto.ca/2011/2011-cannes-declaration-111104-en.html} (accessed 2 February 2019).
    \item    GEOGLAM, 2017. Crop Monitor, \url{https://cropmonitor.org/} (accessed 2 February 2019)
    \item    Iizumi, T., Shin, Y., Kim, W., Kim, M., \& Choi, J. (2018). Global crop yield forecasting using seasonal climate information from a multi‐model ensemble. Climate Services, 11, 13–23. 
\end{enumerate}



\end{document}

